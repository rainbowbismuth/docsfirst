\documentclass{article}
\usepackage{minted}
\usepackage[left=0.85in, right=1.75in]{geometry}
\usepackage{hyperref}
\usepackage{alltt}

\usepackage[most]{tcolorbox}
\definecolor{block-gray}{gray}{0.98}
\newtcolorbox{defquote}{colback=block-gray,grow to right by=20mm,grow to left by=0mm,
boxrule=0pt,boxsep=5pt,breakable}

\setcounter{secnumdepth}{5}

\title{The docsfirst Book}
\author{Emily A. Bellows}

\begin{document}
\maketitle

\section{Introduction}

\marginpar{WARNING: This program \(and text\) is still heavily in development!
Expect lies and incompleteness.}
This book is the complete documentation for the \emph{docsfirst} program.

\paragraph{What does it do?} \emph{docsfirst} is a tool that analyzes source
code for specially marked blocks of code, and inserts it in various ways
into a .tex file.

\paragraph{Why?} I wanted a modern literate programming tool that didn't require
special tooling on either the \LaTeX\ or the program side. Most of the
literate programming tools I was messing around with had custom file types that
no IDE or Syntax Highlighter would support. I wanted a tool that would use a
programming language's native commenting system, and take out the blocks of
code, rather then taking documentation and trying to extract files of code from
it.

\paragraph{What's literate programing?} It's an approach to programming
popularized by Donald Knuth. The idea is to intermix
prose and code in a way that best explains the human thoughts and intentions
behind a program, and not simply the low-level concerns of how code is organized
into folders and files. The goal is to give humans a place to start reading your
code, along with explanations every step of the way so that by the end of the
literate program text, they should understand the entire program.

\paragraph{How far along is the project?} Not very atm, just getting started!
But it shouldn't take long now given how quick the python prototype went.

\paragraph{How is this book written?} Using this very tool of course! Along with
a handful of friends like \emph{pdflatex} and \emph{pdf2htmlex}.

\subsection{Copyright}

\paragraph{GNU GPLv3} The entirety of this project's code is under the GNU
General Public License, please see the file \emph{LICENSE} in the root
directory.

%DOCSFIRST Copyright

\paragraph{Attribution-ShareAlike CC BY-SA} The non-code part of this
literate programming text is under this copy-left license, see
\url{https://creativecommons.org/licenses/by-sa/4.0/} for more details.

\section{Usage}

\paragraph{} Gotta fill this out later

\section{The docsfirst Literate Program}

\paragraph{}
The main task in \emph{docsfirst} is breaking files of source code into delimited,
named blocks that are delinated by special comment markers. These source code
blocks end up in the output .tex file when recalled by name in the input .tex
file.

%DOCSFIRST Define Block

\paragraph{}
So what information do we record about a block? Every field here is critical.
Most important is the \emph{Body}, that's a slice of the lines of code that make
up the block. Next we have a \emph{Description}, which is a name for
a\footnote{Actually, one or more blocks. We'll get to that a bit later} block.
\emph{FileName} is the file in which a block came from, and \emph{LineNumber}
is the line in that file that this block starts on. Finally we have a pointer to
a \emph{Language} which contains useful information about which programming
language a block is written in. \emph{Tag} is an optional string that can be
assigned to a block, and \emph{Indentation} records the whitespace before the
BEGIN marker.

\paragraph{}
You can see all of those fields in action in the above code block in this book.
\emph{Define Block} is a description. The filename is \emph{docsfirst.go}, and
thats followed by the line number where that block starts in that file. The code
you see is the block's body, and finally, the reason \emph{docsfirst} knows how
to syntax highlight the code is because of the data stored in the block's
\emph{Language} field.

\subsection{The ParseBlocks function}

\paragraph{}
The \emph{ParseBlocks} function is one of the two most important functions in
\emph{docsfirst}. It has the job of parsing a file and splitting it up into
blocks.

%DOCSFIRST (ParseBlocks)

This function uses an idiom used heavily elsewhere, where the function creates an
output channel, and spawns a goroutine that puts data onto that output channel.
It then returns immediately. The goroutine that is spawned also closes that
channel at the very end, so that the consumer knows there won't be any more
input.

\paragraph{}
So we have a few variables:

%DOCSFIRST Initialize block parsing state

\emph{lineNumber} is the current line we're on. \emph{beginRegex} and
\emph{endRegex} are two regular expressions that we use to parse BEGIN and END
markers. The rest of them all deal with the state of the current block we are
parsing. \emph{curDescription} is the most central to the control flow of the
function, if it's not an empty string that means we're currently parsing a block.
If it is an empty string, then it means we're not in the middle of a block and
shouldn't touch the rest of block variables.

\paragraph{}
Lets start looping over lines.

%DOCSFIRST Start parsing blocks line by line

\subsubsection{When currently parsing a block}

\paragraph{}
Lets pretend we were already parsing another block already, what happens when we
encounter a new BEGIN marker?

We output the block we were working on, and reset the current block state based
on the new marker. You might ask why I did it this way, personally I felt having
to end every single block with an entire line dedicated to an END marker would
just annoying.

%DOCSFIRST Parsing a block and we find the start of another

So what is this regex anyways? Let's take a look at the constants we use.

%DOCSFIRST Regex Constants

We end up with three groups. The first group in the regex records
whitespace\footnote{This can be safely ignored for now.}. The second is an
optional tag, and finally the rest of the line ends up being the block's
description.

in \emph{ParseBlocks} we set what is in each group to its corresponding variable,
to be used when this new block is completed.

\paragraph{}
Now if we find an END marker instead, we do nearly the same thing, except we
don't start a new block by setting \emph{curDescription} to nil.

%DOCSFIRST Parsing a block and we find an end block marker

\paragraph{}
Now, if we don't match any marker at all, but we are currently parsing a block,
we simply add the line of source code to \emph{curBody}

%DOCSFIRST If no match, append line to body

\subsubsection{When not in the middle of a block}

\paragraph{}
Once again we test to see if we found a BEGIN marker, unlike last time we don't
have a previous block to output because we weren't in the middle of parsing one.
Otherwise the code is the same.

%DOCSFIRST Start parsing a new block

\subsubsection{Handling user errors}

\paragraph{}
There are two main user errors we can catch easily, the first is a dangling END.
This happens when we encounter an END and we're not parsing a block.

%DOCSFIRST Handle dangling block ends

\paragraph{}
Another kind of user error that can occur is reaching the end of a file while in
the middle of parsing a block.

%DOCSFIRST Check if the file ended while parsing a block

\end{document}
