\documentclass{article}
\usepackage{minted}
\usepackage[left=0.85in, right=1.75in]{geometry}
\usepackage{hyperref}

\usepackage[most]{tcolorbox}
\definecolor{block-gray}{gray}{0.98}
\newtcolorbox{defquote}{colback=block-gray,grow to right by=20mm,grow to left by=0mm,
boxrule=0pt,boxsep=5pt,breakable}

\title{The docsfirst Book}
\author{Emily A. Bellows}

\begin{document}
\maketitle

\section{Introduction}

\marginpar{WARNING: This program \(and text\) is still heavily in development!
Expect lies and incompleteness.}
This book is the complete documentation for the \emph{docsfirst} program.

\paragraph{What does it do?} \emph{docsfirst} is a tool that analyzes source
code for specially marked blocks of code, and inserts it in various ways
into a .tex file.

\paragraph{Why?} I wanted a modern literate programming tool that didn't require
special tooling on either the \LaTeX\ or the program side. Most of the
literate programming tools I was messing around with had custom file types that
no IDE or Syntax Highlighter would support. I wanted a tool that would use a
programming language's native commenting system, and take out the blocks of
code, rather then taking documentation and trying to extract files of code from
it.

\paragraph{What's literate programing?} It's an approach to programming
popularized by Donald Knuth. The idea is to intermix
prose and code in a way that best explains the human thoughts and intentions
behind a program, and not simply the low-level concerns of how code is organized
into folders and files. The goal is to give humans a place to start reading your
code, along with explanations every step of the way so that by the end of the
literate program text, they should understand the entire program.

\paragraph{How far along is the project?} Not very atm, just getting started!
But it shouldn't take long now given how quick the python prototype went.

\paragraph{How is this book written?} Using this very tool of course! Along with
a handful of friends like \emph{pdflatex} and \emph{pdf2htmlex}.

\subsection{Copyright}

\paragraph{GNU GPLv3} The entirety of this project's code is under the GNU
General Public License, please see the file \emph{LICENSE} in the root
directory.

%DOCSFIRST Copyright

\paragraph{Attribution-ShareAlike CC BY-SA} The non-code part of this
literate programming text is under this copy-left license, see
\url{https://creativecommons.org/licenses/by-sa/4.0/} for more details.

\section{Usage}

Gotta fill this out later

\section{The docsfirst Literate Program}

The main task in \emph{docsfirst} is breaking files of source code into delimited,
named blocks that are delinated by special comment markers. These source code
blocks end up in the output .tex file when recalled by name in the input .tex
file.

%DOCSFIRST Define Block

So what information do we record about a block? Every field here is critical.
Most important is the \emph{Body}, that's a slice of the lines of code that make
up the block. Next we have a \emph{Description}, which is a name for
a\footnote{Actually, one or more blocks. We'll get to that a bit later} block.
\emph{FileName} is the file in which a block came from, and \emph{LineNumber}
is the line in that file that this block starts on. Finally we have a pointer to
a \emph{Language} which contains useful information about which programming
language a block is written in.

You can see all of those fields in action in the above code block in this book.
\emph{Define Block} is a description. The filename is \emph{docsfirst.go}, and
thats followed by the line number where that block starts in that file. The code
you see is the block's body, and finally, the reason \emph{docsfirst} knows how
to syntax highlight the code is because of the data stored in the block's
\emph{Language} field.

\end{document}
